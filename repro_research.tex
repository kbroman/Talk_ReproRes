\documentclass[12pt,t]{beamer}
\usepackage{graphicx}
\setbeameroption{hide notes}
\setbeamertemplate{note page}[plain]
\usepackage{listings}

% header.tex: boring LaTeX/Beamer details + macros

% get rid of junk
\usetheme{default}
\beamertemplatenavigationsymbolsempty
\hypersetup{pdfpagemode=UseNone} % don't show bookmarks on initial view


% font
\usepackage{fontspec}
\setsansfont
  [ ExternalLocation = fonts/ ,
    UprightFont = *-regular ,
    BoldFont = *-bold ,
    ItalicFont = *-italic ,
    BoldItalicFont = *-bolditalic ]{texgyreheros}
\setbeamerfont{note page}{family*=pplx,size=\footnotesize} % Palatino for notes
% "TeX Gyre Heros can be used as a replacement for Helvetica"
% I've placed them in fonts/; alternatively you can install them
% permanently on your system as follows:
%     Download http://www.gust.org.pl/projects/e-foundry/tex-gyre/heros/qhv2.004otf.zip
%     In Unix, unzip it into ~/.fonts
%     In Mac, unzip it, double-click the .otf files, and install using "FontBook"

% named colors
\definecolor{offwhite}{RGB}{255,250,240}
\definecolor{gray}{RGB}{155,155,155}
\definecolor{purple}{RGB}{177,13,201}
\definecolor{green}{RGB}{46,204,64}

\definecolor{background}{RGB}{255,255,255}
\definecolor{foreground}{RGB}{24,24,24}
\definecolor{title}{RGB}{27,94,134}
\definecolor{subtitle}{RGB}{22,175,124}
\definecolor{hilit}{RGB}{122,0,128}
\definecolor{vhilit}{RGB}{255,0,128}
\definecolor{codehilit}{RGB}{255,0,128}
\definecolor{lolit}{RGB}{95,95,95}
\definecolor{myyellow}{rgb}{1,1,0.7}
\definecolor{nhilit}{RGB}{128,0,128}  % hilit color in notes
\definecolor{nvhilit}{RGB}{255,0,128} % vhilit for notes

\newcommand{\hilit}{\color{hilit}}
\newcommand{\vhilit}{\color{vhilit}}
\newcommand{\nhilit}{\color{nhilit}}
\newcommand{\nvhilit}{\color{nvhilit}}
\newcommand{\lolit}{\color{lolit}}

% use those colors
\setbeamercolor{titlelike}{fg=title}
\setbeamercolor{subtitle}{fg=subtitle}
\setbeamercolor{institute}{fg=lolit}
\setbeamercolor{normal text}{fg=foreground,bg=background}
\setbeamercolor{item}{fg=foreground} % color of bullets
\setbeamercolor{subitem}{fg=lolit}
\setbeamercolor{itemize/enumerate subbody}{fg=lolit}
\setbeamertemplate{itemize subitem}{{\textendash}}
\setbeamerfont{itemize/enumerate subbody}{size=\footnotesize}
\setbeamerfont{itemize/enumerate subitem}{size=\footnotesize}

% page number
\setbeamertemplate{footline}{%
    \raisebox{5pt}{\makebox[\paperwidth]{\hfill\makebox[20pt]{\lolit
          \scriptsize\insertframenumber}}}\hspace*{5pt}}

% add a bit of space at the top of the notes page
\addtobeamertemplate{note page}{\setlength{\parskip}{12pt}}

% default link color
\hypersetup{colorlinks, urlcolor={hilit}}

\lstset{language=bash,
        basicstyle=\ttfamily\scriptsize,
        frame=single,
        commentstyle=,
        backgroundcolor=\color{offwhite},
        showspaces=false,
        showstringspaces=false
        }


% a few macros
\newcommand{\bi}{\begin{itemize}}
\newcommand{\bbi}{\vspace{24pt} \begin{itemize} \itemsep8pt}
\newcommand{\ei}{\end{itemize}}
\newcommand{\be}{\begin{enumerate}}
\newcommand{\bbe}{\vspace{24pt} \begin{enumerate} \itemsep8pt}
\newcommand{\ee}{\end{enumerate}}
\newcommand{\sbi}{\begin{itemize} \fontsize{9pt}{9.5}\selectfont}
\newcommand{\sbe}{\begin{enumerate} \fontsize{9pt}{9.5}\selectfont}
\newcommand{\ig}{\includegraphics}
\newcommand{\subt}[1]{{\footnotesize \color{subtitle} {#1}}}
\newcommand{\ttsm}{\tt \small}
\newcommand{\ttfn}{\tt \footnotesize}
\newcommand{\figh}[2]{\centerline{\includegraphics[height=#2\textheight]{#1}}}
\newcommand{\figw}[2]{\centerline{\includegraphics[width=#2\textwidth]{#1}}}


%%%%%%%%%%%%%%%%%%%%%%%%%%%%%%%%%%%%%%%%%%%%%%%%%%%%%%%%%%%%%%%%%%%%%%
% end of header
%%%%%%%%%%%%%%%%%%%%%%%%%%%%%%%%%%%%%%%%%%%%%%%%%%%%%%%%%%%%%%%%%%%%%%

% title info
\title{Reproducible Research}
\author{\href{http://kbroman.org}{Karl Broman}}
\institute{Biostatistics \& Medical Informatics, UW{\textendash}Madison}
\date{\href{http://kbroman.org}{\tt \scriptsize \color{foreground} kbroman.org}
\\[-4pt]
\href{https://github.com/kbroman}{\tt \scriptsize \color{foreground} github.com/kbroman}
\\[-4pt]
\href{https://twitter.com/kwbroman}{\tt \scriptsize \color{foreground} @kwbroman}
\\[2pt]
\scriptsize {\lolit Slides:} \href{http://bit.ly/Broman2015-05}{\tt \scriptsize
  \color{foreground} bit.ly/Broman2015-05}
}


\begin{document}

% title slide
{
\setbeamertemplate{footline}{} % no page number here
\frame{
  \titlepage

  \vfill \hfill \includegraphics[height=6mm]{Figs/cc-zero.png} \vspace*{-1cm}

  \note{These are slides for a talk I will give for the Molecular
    Microbial Ecology and Evolution (MoMiEE) group at UW-Madison on 11
    May 2015.

    Source: {\tt https://github.com/kbroman/Talk\_ReproRes} \\
    Slides: {\tt http://bit.ly/Broman2015-05\_nonotes} \\
    With notes: {\tt http://bit.ly/Broman2015-05}

}
} }


\begin{frame}[fragile,c]{}

\begin{center}
\begin{minipage}[c]{9.3cm}
\begin{semiverbatim}
\lstset{basicstyle=\normalsize}
\begin{lstlisting}[linewidth=9.3cm]
 Karl -- this is very interesting,
 however you used an old version of
 the data (n=143 rather than n=226).

 I'm really sorry you did all that
 work on the incomplete dataset.

 Bruce
\end{lstlisting}
\end{semiverbatim}
\end{minipage}
\end{center}

\note{This is an edited version of an email I got from a collaborator,
  in response to an analysis report that I had sent him.

  I try to always include some brief data summaries at the start of
  such reports. By doing so, he immediately saw that I had an old
  version of the data.

  Because I'd set things up carefully, I could just substitute in the
  newer dataset, type ``{\tt make}'', and get the revised report.

  This is a reproducibility success story. But it took me a long
  time to get to this point.
}
\end{frame}


\begin{frame}[c]{}
\centerline{\Large In what order do I run these scripts?}

\note{Let me start with some examples of reproducibility gone wrong.

  Sometimes the process of data file manipulation and data
  cleaning gets spread across a bunch of scripts that need to be
  executed in a particular order. Will I record this information? Is
  it obvious what script does what?}
\end{frame}



\begin{frame}[c]{}
\centerline{\Large Where did we get this data file?}

\note{Record the provenance of all data or metadata files.}
\end{frame}



\begin{frame}[c]{}
\centerline{\Large Why did I omit those samples?}

\note{I may decide to omit a few samples. Will I record {\nhilit why}
  I omitted those particular samples?}
\end{frame}



\begin{frame}[c]{}
\centerline{\Large How did I make that figure?}

\note{Sometimes, in the midst of a bout of exploratory data analysis,
  I'll create some exciting graph and have a heck of a time
  reproducing it afterwards.}
\end{frame}



\begin{frame}[c]{}
\centerline{\Large "Your script is now giving an error."}

\note{It was working last week. Well, last month, at least.

How easy is it to go back through that script's history to see where
and why it stopped working?}
\end{frame}



\begin{frame}[c]{}
\centerline{\Large "The attached is similar to the code we used."}

\note{From an email in response to my request for code used for a
  paper.}
\end{frame}



\begin{frame}[c]{}


\centering
\Large

Reproducible

\bigskip

\onslide<2->{{\color{lolit} vs.}}

\bigskip

\only<1|handout 0>{{\color{background} (Replicable) invisible text}}
\only<2>{Replicable}
\only<3 | handout 0>{Correct}

\note{Computational work is
  {\color{nhilit} reproducible} if one can take the data and code and produce
  the same set of results. {\color{nhilit} Replicable} is more stringent: can
  someone repeat the experiment and get the same results?

  Reproducibility is a minimal standard. That something is
  reproducible doesn't imply that it is correct. The code may have bugs. The
  methods may be poorly behaved. There could be experimental
  artifacts.

  (But reproducibility is probably associated with correctness.)

  Note that some scientists say replicable for what I call
  reproducible, and vice versa.
}
\end{frame}



\begin{frame}{Levels of quality}


\vspace{24pt}

\bi
\itemsep12pt
\item Are the tables and figures reproducible from the code and data?
\item Does the code actually do what you think it does?
\item In addition to {\color{hilit} what} was done, is it clear
  {\color{hilit} why} it was done?
  \bi
  \item[] (e.g., how were parameter settings chosen?)
  \ei
\item Can the code be used for other data?
\item Can you extend the code to do other things?
\ei

\note{Reproducibility is not black and white. And the ideal is hard to
  achieve.
}
\end{frame}


\begin{frame}{Why do we care?}

\vspace{24pt}

\bi
\itemsep12pt
\item Avoid embarrassment
\item More likely correct
\item Save time, in the long run
\item Greater potential for extensions; higher impact
\ei

\note{Doing things properly (writing clear, documented, well-tested
  code) is time consuming, but it could save you a ton of aggravation
  down the road.  Ultimately, you'll be more efficient, and your work
  will have greater impact.

Your code and analyses will be easier to debug, maintain, and extend.
}
\end{frame}


\begin{frame}[c]{Steps toward reproducible research}

  \centering
  \Large

  \href{http://kbroman.org/steps2rr}{\tt \color{foreground} kbroman.org/steps2rr}

\note{The above website contains my thoughts on how to towards full
  reproducibility.

  Don't try to change every aspect of your workflow all at once.
}
\end{frame}


\begin{frame}[c]{1. Everything with a script}

\centering
\large
If you do something once, \\
you'll do it 1000 times.

\note{The most basic principle for reproducible research is: do
  everything via code.

  Downloading data from the web, converting an Excel file to CSV,
  renaming columns/variables, omitting bad samples or data points...do
  all of this with scripts.

  You may be tempted to open up a data file and hand-edit. But if you
  get a revised version of that file, you'll need to do it again. And
  it'll be harder to figure out what it was that you did.

  Some things are more cumbersome via code, but in the long run you'll
  save time.
}
\end{frame}




\begin{frame}[fragile,c]{2. Organize your data \& code}

\begin{center}
\begin{minipage}[c]{10.3cm}
\begin{semiverbatim}
\lstset{basicstyle=\normalsize}
\begin{lstlisting}[linewidth=10.3cm]
RawData/              ReadMe.txt
DataSummary/          ToDo.txt
DerivedData/          Makefile

Python/               Notes/
R/                    Refs/
Ruby/
\end{lstlisting}
\end{semiverbatim}
\end{minipage}
\end{center}

\note{You don't {\nhilit need} to be organized, but it sure will help
  others (or yourself, later), when you try to figure out what it was
  that you did.

  Segregate all the materials for a project in one directory/folder on
  your harddrive.

  I prefer to separate raw data from processed data, and I put code in
  a separate file.

  Write {\tt ReadMe} files to explain what's what.
}
\end{frame}


\begin{frame}[c]{}

\centering
\large
Your closest collaborator is you six months ago, but you
don't reply to emails.
\note{I heard this from Paul Wilson, UW-Madison.
}
\end{frame}



\begin{frame}[fragile,c]{3. Automate the process (GNU Make)}

\begin{semiverbatim}
\begin{lstlisting}
rqtlexper.pdf: rqtlexper.tex rqtlexper.bib fig1.pdf fig2.pdf
    pdflatex rqtlexper
    bibtex rqtlexper
    pdflatex rqtlexper
    pdflatex rqtlexper

rqtlexper.tex: rqtlexper.Rnw Data/lines_code_detail.txt
    R -e 'library(knitr);knit("rqtlexper.Rnw")'

fig1.pdf: R/create_fig1.R
    cd R;R CMD BATCH create_fig1.R

fig2.pdf: R/create_fig2.R Data/nlines_by_ver.csv
    cd R;R CMD BATCH rqtl_lines_code.R

Data/nlines_by_ver.csv: Python/grab_lines.py Data/versions.txt
    Python/grab_lines_code.py
\end{lstlisting}
\end{semiverbatim}

\note{GNU Make is an old (and rather quirky) tool for automating the
  process of building computer programs. But it's useful much more
  broadly, and I find it valuable for automating the full process of
  data file manipulation, data cleaning, and analysis.

  In addition to {\nhilit automating} a complex process, it also
  {\nhilit documents} the process, including the dependencies among
  data files and scripts.
}
\end{frame}



\begin{frame}[c]{4. Turn scripts into reproducible reports}


\vspace*{8mm}

\figw{Figs/example_Rmd.png}{0.92}
\onslide<2>{
  \vspace*{-0.45\textheight}
  \hspace*{0.10\textwidth}
  \figw{Figs/example_Rmd_source.png}{0.92}
}

\note{I {\nhilit love} R Markdown for making reproducible reports that
  document the full details of my analysis. R Markdown mixes Markdown
  (for light-weight markup of text) and R code chunks; when processed
  with knitr, the R code is executed and results inserted into the
  final document.

  With these informal reports, I seek to fully capture the entirety of
  my data explorations and decisions.

  Python people should look at iPython notebooks.
}
\end{frame}


\begin{frame}[fragile,c]{5. Turn repeated code into functions}

\begin{semiverbatim}
\begin{lstlisting}
# Python
def read_genotypes (filename):
    "Read matrix of genotype data"
\end{lstlisting}
\end{semiverbatim}

\vspace*{-8mm}

\begin{semiverbatim}
\begin{lstlisting}
# R
plot_genotypes <-
function(genotypes, ...)
{
}
\end{lstlisting}
\end{semiverbatim}

\note{Pull out complex or repeated code as a separate function.
  This makes your code easier to read and maintain.}
\end{frame}


\begin{frame}[c]{6. Create a package/module}

  \centering
  \Large
  {\hilit DRY}: Don't repeat yourself

  \note{It's surprisingly easy to create an R package (see
    {\tt http://kbroman.org/pkg\_primer}) and it's even easier to make
    a Python module.

    When writing functions, try to write them in a somewhat-general
    way and then pull them out of the project as separate package or
    module, so that you (and/or others) may reuse them for other purposes.
}
\end{frame}


\begin{frame}<handout:0>[c]{7. Use version control (git/GitHub)}

\vspace*{3mm}

\centering

% comic from http://www.phdcomics.com/comics/archive.php?comicid=1531
\only<1>{\figh{Figs/phd101212s.png}{0.8}}

\end{frame}


\begin{frame}<handout:0>[c]{7. Use version control (git/GitHub)}

\only<1>{\addtocounter{framenumber}{-1}}

\centering

\only<1-2>{\figh{Figs/example_repo}{0.80}}
\onslide<2>{
  \vspace{-0.65\textheight}
  \figh{Figs/example_repo_zoom}{0.55}
}
\end{frame}


\begin{frame}[c]{7. Use version control (git/GitHub)}

\only<1>{\addtocounter{framenumber}{-1}}

\vspace*{3mm}

\centering

\only<1|handout 0>{\figh{Figs/example_history}{0.80}}
\only<2>{\figh{Figs/example_commit}{0.80}}

\note{
  git has a steep learning curve, but ultimately I think you'll find
  it really helpful.

  The big selling point is in collaboration: merging changes from
  collaborators, and keep your work synchronized.

  Longer term, there's great value in having the entire history of
  changes to your project. If something stops working, you can go
  back to any point in that history to see when it stopped working and
  why.

  With git, you can also work on new features or analyses without fear
  of breaking the parts that are currently working well.
}
\end{frame}



\begin{frame}{8. License your software}

\vspace{60pt}

\centerline{\large Pick a license, any license}

\vspace{18pt}

\hfill
{\textendash} \href{http://blog.codinghorror.com/pick-a-license-any-license/}{Jeff Atwood}

\note{
  If you don't pick a license for your software, no one else can use it.

  So if you want to distribute your code so that others can reproduce
  your analyses, you need to pick a license, any license.

  I choose between the MIT license and the GPL.

  Don't use the Creative Commons licenses for code. But feel free to
  use them for other things.
}
\end{frame}


\begin{frame}[c]{Summary}

  \begin{enumerate}
  \itemsep12pt
  \item Everything with a script
  \item Organize your data \& code
  \item Automate the process (GNU Make)
  \item Turn scripts into reproducible reports
  \item Turn repeated code into functions
  \item Create a package/module
  \item Use version control (git/GitHub)
  \item Pick a license, any license
  \end{enumerate}

  \note{
    It's always good to include a summary.
}
\end{frame}

\begin{frame}[c]{}

\Large

Slides: \href{http://bit.ly/Broman2015-05}{\tt bit.ly/Broman2015-05} \quad
\includegraphics[height=5mm]{Figs/cc-zero.png}

\vspace{10mm}

\href{http://kbroman.org}{\tt kbroman.org}

\vspace{10mm}

\href{https://github.com/kbroman}{\tt github.com/kbroman}

\vspace{10mm}

\href{https://twitter.com/kwbroman}{\tt @kwbroman}


\note{
  Here's where you can find me, as well as the slides for this talk.
}
\end{frame}




\end{document}
